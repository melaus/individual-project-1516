\documentclass[11pt,openright,a4paper]{report}
\usepackage{graphicx}
\usepackage{pdflscape}
\usepackage{hyperref}
\usepackage{titlesec}
\usepackage{geometry}
\graphicspath{ {images/} }

\geometry{
  a4paper,
  top=20mm,
  bottom=20mm
}

% title
\title{
  Using Machine-learning Techniques to Increase the Efficiency of Kinect Fusion Reconstructions
  \\~\\
  \textbf{Literature Review}
}
\author{Alan Lau}
\date{MComp Computer Science\\University of Bath\\October 2015}

% tab spacing
\newcommand{\tab}[1]{\hspace{.2\textwidth}\rlap{#1}}

% set chapter format
\titleformat{\chapter}[hang]{\LARGE\bfseries}{\thechapter\hspace{20pt}}{0pt}{\LARGE\bfseries}
\titlespacing{\chapter}{0pt}{0pt}{5pt}

% set section format
\titleformat{\section}[hang]{\Large\bfseries}{\thesection\hspace{20pt}}{0pt}{\Large\bfseries}
\titlespacing{\section}{0pt}{0pt}{-3pt}


%%%%%%%%%%%%%%%%%%%%%%%%%%%%%%%%%%%%


\begin{document}

% title page
\maketitle
\newpage

% ToC
\tableofcontents
\newpage

% reset page number
\setcounter{page}{1}

% set pagragraph style
\setlength{\parindent}{0pt}
\setlength{\parskip}{\baselineskip}

% introduction 
\chapter{Introduction}
This project aims to quicken the time required to create a 3-diemensional reconstruction of objects in a room with just a simple 2-dimentional scan from one angle, using a Microsoft Kinect Camera. To enable such goal, classification, a form of machine-learning, is required. There are many different statistical classification algorithms that can be used for labelling objects through training with similar objects and recognising the category and object belongs when in use. 

A group of New York University researchers have created a depth dataset named the NYU Depth Dataset. A variety of objects are scanned or segmented from a scene. The large number of scans for a class of objects makes it ideal to be used as training data for the classifier. By understanding more about this dataset and how it has been used, it provides a 

This Review tries to explore more about depth maps and the depth dataset, understand how these algorithm fit in with depth data and .

\newpage


% Kinect Fusion
\chapter{Kinect Fusion}
The Kinect Camera is an RGBD Camera. It has an infrared-based depth camera on top of a conventional coloured camera. This enables extra information about how far an object is from the camera, so an accurate reconstruction is possible.
\section{Depth Map}
\section{NYU Depth Dataset V2}
\newpage


% classifiers
\chapter{Classifiers}
\section{Classification Algorithms}
\section{Using Classification Alogorithms for Depth Data}
% Integrating Kinect Depth Data with a Stochastic Object Classification Framework for Forestry Robot
% "shopping list table"
\newpage

% displaying the results
\chapter{Displaying the Results}
\subsection{OpenGL}


% summary
\chapter{Summary}
\newpage

% References
\begin{thebibliography}{7}
  \bibitem{msdn-kinect-info}
    Microsoft Corporation.
    \textit{Kinect Fusion} [Online]. 
    Place of publication: Microsoft Developer Network.
    Available from: \url{https://msdn.microsoft.com/en-us/library/dn188670.aspx}
    [Accessed 22 October 2015].

  \bibitem{blog-kinect-annc}
    Kinect for Windows Team, 2013.
    \textit{Kinect Fusion demonstrated at Microsoft Research TechFest, coming soon to SDK} [Online].
    Place of publication: Kinect for Windows Blog
    Available from: \url{http://blogs.msdn.com/b/kinectforwindows/archive/2013/03/06/kinect-fusion-demonstrated-at-microsoft-research-techfest-coming-soon-to-sdk.aspx}
    [Accessed 22 October 2015].

  \bibitem{blog-semantic-paint}
    Cameron, S., 2015.
    \textit{Microsoft Research debuts another project, Semantic Paint} [Online].
    Place of publication: WinBeta Blog
    Available from: \url{http://www.winbeta.org/news/microsoft-research-debuts-another-project-semantic-paint}
    [Accessed 23 October 2015].

  \bibitem{art-DyanmicFusion}
    Newcombe, R. A. et al., 2015.
    DynamicFusion: Reconstruction and Tracking of Non-rigid Scenes in Real-Time. 
    \textit{Proceedings of the IEEE Conference on Computer Vision and Pattern Recognition}, 7-12 June 2015 Boston, Massachusetts.
    Seatle: University of Washington, pp.343-352.
    %Available from: \url{http://homes.cs.washington.edu/~newcombe/papers/DynamicFusion.pdf}
    %[Accessed 23 October 2015].

  \bibitem{art-dense-surface}
    Newcombe, R A et al., 2011.
    KinectFusion: Real-time dense surface mapping and tracking.
    \textit{Mixed and augmented reality (ISMAR), 2011 10th IEEE international symposium}, 26-29 October 2011 Basel, Switzerland.
    Seatle: University of Washington, pp.127-136
    %Available from: \url{http://homes.cs.washington.edu/~newcombe/papers/newcombe_etal_ismar2011.pdf}
    %[Accessed 23 October 2015].

  \bibitem{the-HouseScan}
    Hambüchen, N., 2014.
    \textit{HouseScan: Building-scale interior 3D reconstruction with KinectFusion}. 
    Thesis (MSc).
    Imperial College, London.
    %Available from: \url{http://www.doc.ic.ac.uk/teaching/distinguished-projects/2014/n.hambuechen.pdf}
    %[Accessed 23 October 2015].

  \bibitem{}
    Silberman, N. et al..
    \textit{NYU Depth Dataset V2} [Online].
    Available from: \url{http://cs.nyu.edu/~silberman/datasets/nyu_depth_v2.html}
    [Accessed 23 October 2015].

\end{thebibliography}
\newpage


\end{document}
